
\label{sec:subsection61}
Рассмотрим канал, состоящий из $N$ контрольных объёмов, пронумерованных от $0$ до $N-1$. Слева канал ограничен входным узлом с давлением $P_{in}$, а справа --- выходным узлом с давлением $P_{out}$. Первая гидравлическая связь, связывающая входной узел с первой расчётной ячейкой канала, имеет номер $0$, а последняя гидравлическая связь, связывающая последнюю расчётную ячейку канала с выходным узлом --- номер $N$. Общее количество гидравлических связей --- $N+1$.    

Запишем уравнения сохранения массы и импульса в форме~\eqref{formula518} и~\eqref{formula511} для трёх соседних расчётных ячеек с номерами $j-1$, $j$ и $j+1$. Гидравлические связи между ними будут иметь номера $j$ и $j+1$. Тогда имеем систему из трёх уравнений:
\begin{equation}
\label{formula61}
\left\{
\begin{aligned}
A_j^G \cdot \Delta(dP_{j-1}^n)+B_j^G \cdot \Delta(dG_j^n) + C_j^G \cdot \Delta(dP_j^n) & = -F_j^G \\
\left(A_j^P \right)' \cdot \Delta(dG_j^n) + \left(B_j^P \right)' \cdot \Delta(dP_j^n) + \left(C_j^P \right)' \cdot \Delta(dG_{j+1}^n) & = -\left(F_j^P \right)' \\
A_{j+1}^G \cdot \Delta(dP_j^n)+B_{j+1}^G \cdot \Delta(dG_{j+1}^n) + C_{j+1}^G \cdot \Delta(dP_{j+1}^n) & = -F_{j+1}^G.
\end{aligned}
\right.
\end{equation}

Выразим из первого и третьего уравнения расходы в гидравлических связях через давления в ячейках:
\begin{equation}
\label{formula62}
\left\{
\begin{aligned}
\Delta(dG_j^n) & = -\frac{F_j^G+A_j^G \cdot \Delta(dP_{j-1}^n)+C_j^G \cdot \Delta(dP_j^n)}{B_j^G} \\
\Delta(dG_{j+1}^n) & = -\frac{F_{j+1}^G+A_{j+1}^G \cdot \Delta(dP_j^n)+C_{j+1}^G \cdot \Delta(dP_{j+1}^n)}{B_{j+1}^G}.
\end{aligned}
\right.
\end{equation}

Подставим полученные выражения в уравнение сохранения массы для $j$-й ячейки:
\begin{eqnarray}
\label{formula63}
-\frac{\left(A_j^P \right)'}{B_j^G}
\cdot \left(F_j^G+A_j^G \cdot \Delta(dP_{j-1}^n)+C_j^G \cdot \Delta(dP_j^n)\right) + \left(B_j^P \right)' \cdot \Delta(dP_j^n) - \nonumber ~\\
-\frac{\left(C_j^P \right)'}{B_{j+1}^G} \cdot \left(F_{j+1}^G+A_{j+1}^G \cdot \Delta(dP_j^n)+C_{j+1}^G \cdot \Delta(dP_{j+1}^n) \right) = -\left(F_j^P \right)'.
\end{eqnarray}

Раскроем скобки и приведём подобные члены перед давлениями. Получим
\begin{align}
\label{formula64}
&\left(-\left(A_j^P \right)' \cdot \frac{A_j^G}{B_j^G} \right) \cdot \Delta(dP_{j-1}^n)+\left(-\left(A_j^P \right)' \cdot \frac{C_j^G}{B_j^G} + \left(B_j^P \right)' - \left(C_j^P \right)'\cdot \frac{A_{j+1}^G}{B_{j+1}^G} \right)\cdot \Delta(dP_j^n) + \notag ~\\
+&\left(- \left(C_j^P \right)'\cdot \frac{C_{j+1}^G}{B_{j+1}^G}   \right)\cdot\Delta(dP_{j+1}^n) = -\left(\left(F_j^P \right)' - \frac{\left(A_j^P \right)'}{B_j^G}\cdot F_j^G - \frac{\left(C_j^P \right)'}{B_{j+1}^G}\cdot F_{j+1}^G \right).   
\end{align}

Переобозначим коэффициенты, поставив два штриха. Окончательно получим уравнение, связывающее давления в трёх соседних ячейках канала:
\begin{equation}
\label{formula65}
\boxed{\left(A_j^P \right)'' \cdot \Delta(dP_{j-1}^n) + \left(B_j^P \right)'' \cdot \Delta(dP_j^n) + \left(C_j^P \right)'' \cdot \Delta(dP_{j+1}^n) = -\left(F_j^P \right)''}.
\end{equation}

Получилось трёхточечное уравнение. Система таких уравнений эффективно решается методом прогонки. Предположим, что неизвестные приращения давлений связаны рекуррентным соотношением
\begin{equation}
\label{formula66}
\Delta(dP_j^n)=\beta_{j+1} \cdot \Delta(dP_{j+1}^n) + \alpha_{j+1}.
\end{equation}

Используя это соотношение, выразим $\Delta(dP_{j-1}^n)$ и $\Delta(dP_j^n)$ через $\Delta(dP_{j+1}^n)$:
\begin{equation}
\label{formula67}
\left\{
\begin{aligned}
\Delta(dP_j^n) & = \beta_{j+1} \cdot \Delta(dP_{j+1}^n) + \alpha_{j+1} \\
\Delta(dP_{j-1}^n) & = \beta_j \cdot \Delta(dP_j^n) + \alpha_j = \beta_j \cdot (\beta_{j+1} \cdot \Delta(dP_{j+1}^n) + \alpha_{j+1}) + \alpha_j.
\end{aligned}
\right.
\end{equation}

Подставим~\eqref{formula67} в~\eqref{formula65}. Получим
\begin{eqnarray}
\label{formula68}
\left(A_j^P \right)'' \cdot (\beta_j \cdot \beta_{j+1} \cdot \Delta(dP_{j+1}^n) + \beta_j \cdot \alpha_{j+1} + \alpha_j ) + \left(B_j^P \right)'' \cdot (\beta_{j+1} \cdot \Delta(dP_{j+1}^n) + \alpha_{j+1}) + \nonumber ~\\
+ \left(C_j^P \right)'' \cdot \Delta(dP_{j+1}^n) = -\left(F_j^P \right)'';  
\end{eqnarray}
\begin{eqnarray}
\label{formula69}
\left(\left(A_j^P \right)'' \cdot \beta_j \cdot \beta_{j+1} + \left(B_j^P \right)'' \cdot \beta_{j+1} + \left(C_j^P \right)'' \right)\cdot\Delta(dP_{j+1}^n)+ \nonumber ~\\
+\left(\left(A_j^P \right)'' \cdot \beta_j \cdot \alpha_{j+1} + \left(A_j^P \right)'' \cdot \alpha_j + \left(B_j^P \right)'' \cdot \alpha_{j+1} + \left(F_j^P \right)'' \right)=0.  
\end{eqnarray}

Чтобы это равенство выполнялось независимо от решения, необходимо, чтобы удовлетворялись следующие равенства:
\begin{equation}
\label{formula610}
\left\{
\begin{aligned}
\left(A_j^P \right)'' \cdot \beta_j \cdot \beta_{j+1} + \left(B_j^P \right)'' \cdot \beta_{j+1} + \left(C_j^P \right)'' & = 0 \\
\left(A_j^P \right)'' \cdot \beta_j \cdot \alpha_{j+1} + \left(A_j^P \right)'' \cdot \alpha_j + \left(B_j^P \right)'' \cdot \alpha_{j+1} + \left(F_j^P \right)'' & = 0.
\end{aligned}
\right.
\end{equation}

Отсюда следуют рекуррентные соотношения для нахождения прогоночных коэффициентов:
\begin{equation}
\label{formula611}
\left\{
\begin{aligned}
\beta_{j+1} & = \frac{-\left(C_j^P \right)''}{\left(A_j^P \right)'' \cdot \beta_j + \left(B_j^P \right)''}; \\
\alpha_{j+1} & = \frac{-\left(F_j^P \right)''-\left(A_j^P \right)'' \cdot \alpha_j}{\left(A_j^P \right)'' \cdot \beta_j + \left(B_j^P \right)''}.
\end{aligned}
\right.
\end{equation}

Из уравнения для первой ячейки канала находим
\begin{eqnarray}
\label{formula612}
\left(A_0^P \right)'' \cdot \Delta(dP_{in}^n) + \left(B_0^P \right)'' \cdot \Delta(dP_0^n) + \left(C_0^P \right)'' \cdot \Delta(dP_1^n) = -\left(F_0^P \right)''; \nonumber ~\\
\Delta(dP_0^n)=-\frac{\left(C_0^P \right)''}{\left(B_0^P \right)''} \cdot \Delta(dP_1^n) + \frac{-\left(F_0^P \right)'' - \left(A_0^P \right)'' \cdot \Delta(dP_{in}^n)}{\left(B_0^P \right)''}, 
\end{eqnarray}
но $\Delta(dP_0^n)=\beta_1 \cdot \Delta(dP_1^n) + \alpha_1$, поэтому получаем для первых прогоночных коэффициентов выражения
\begin{equation}
\label{formula613}
\left\{
\begin{aligned}
\beta_1 & =  -\frac{\left(C_0^P \right)''}{\left(B_0^P \right)''}; \\
\alpha_1 & = -\frac{\left(F_0^P \right)'' + \left(A_0^P \right)'' \cdot \Delta(dP_{in}^n)}{\left(B_0^P \right)''}.
\end{aligned}
\right.
\end{equation}

Выразим прогоночные коэффициенты $\alpha_j$ через давление во входном узле. Можно показать справедливость следующего тождества:
\begin{equation}
\label{formula614}
\alpha_{j+1}=\omega_j \cdot \Delta(dP_{in}^n) + u_j,
\end{equation}
где коэффициенты $\omega_j$ и $u_j$ находятся из следующих рекуррентных соотношений
\begin{equation}
\label{formula615}
\left\{
\begin{aligned}
\omega_{j+1} & = \frac{\left(A_{j+1}^P \right)'' \cdot \omega_j}{-\left(B_{j+1}^P \right)'' - \left(A_{j+1}^P \right)'' \cdot \beta_{j+1}}; \\
u_{j+1} & = \frac{\left(A_{j+1}^P \right)'' \cdot u_j + \left(F_{j+1}^P \right)'' }{-\left(B_{j+1}^P \right)'' - \left(A_{j+1}^P \right)'' \cdot \beta_{j+1}}, 
\end{aligned}
\right.
\end{equation}
и, кроме того,
\begin{equation}
\label{formula616}
\left\{
\begin{aligned}
\omega_0 & = -\frac{\left(A_0^P \right)''}{\left(B_0^P \right)''}; \\
u_0 & = -\frac{\left(F_0^P \right)''}{\left(B_0^P \right)''}. 
\end{aligned}
\right.
\end{equation}

Выразим давления в ячейках канала через давление в выходном узле, начиная с последней ячейки с номером $N-1$. Для этого используем уравнение~\eqref{formula66}:
\begin{eqnarray}
\label{formula617}
\Delta(dP_{N-1}^n)=\beta_N \cdot \Delta(dP_{out}^n) + \alpha_N; \nonumber ~\\
\Delta(dP_{N-2}^n)=\beta_{N-1} \cdot \Delta(dP_{N-1}^n) + \alpha_{N-1}=\beta_{N-1} \cdot (\beta_N \cdot \Delta(dP_{out}^n) + \alpha_N) + \alpha_{N-1} = \nonumber ~\\
= \beta_{N-1} \cdot \beta_N \cdot \Delta(dP_{out}^n) + (\beta_{N-1} \cdot \alpha_N + \alpha_{N-1}); \nonumber ~\\
\Delta(dP_{N-3}^n)=\beta_{N-2} \cdot \Delta(dP_{N-2}^n) + \alpha_{N-2} = \nonumber ~\\
= \beta_{N-2} \cdot (\beta_{N-1} \cdot \beta_N \cdot \Delta(dP_{out}^n) + (\beta_{N-1} \cdot \alpha_N + \alpha_{N-1})) + \alpha_{N-2} = \nonumber ~\\
= \beta_{N-2} \cdot \beta_{N-1} \cdot \beta_N \cdot \Delta(dP_{out}^n) + (\beta_{N-2} \cdot \beta_{N-1} \cdot \alpha_N + \beta_{N-2} \cdot \alpha_{N-1} + \alpha_{N-2})
\end{eqnarray}
и так далее вплоть до первой ячейки, для которой получим
\begin{equation}
\label{formula618}
\Delta(dP_0^n)=B \cdot \Delta(dP_{out}^n) + A,
\end{equation}
где $B = \beta_N \cdot \beta_{N-1} \cdot \beta_{N-2} \dots \beta_1$; 

\noindent $A = \alpha_N \cdot \beta_{N-1} \cdot \beta_{N-2} \dots \beta_1 + \alpha_{N-1} \cdot \beta_{N-2} \dots \beta_1 + \dots + \alpha_2 \cdot \beta_1 + \alpha_1$. 

Подставим в выражение для $A$ вместо $\alpha_j$ их выражения из уравнения~\eqref{formula614}. Получим
\begin{eqnarray}
\label{formula619}
A=(\omega_{N-1} \cdot \Delta(dP_{in}^n) + u_{N-1}) \cdot \beta_{N-1} \cdot \beta_{N-2} \dots \beta_1 + \nonumber ~\\
+ (\omega_{N-2} \cdot \Delta(dP_{in}^n) + u_{N-2}) \cdot \beta_{N-2} \dots \beta_1 + \dots + \nonumber ~\\
+ (\omega_1 \cdot \Delta(dP_{in}^n) + u_1) \cdot \beta_1 + (\omega_0 \cdot \Delta(dP_{in}^n) + u_0).
\end{eqnarray} 

Раскроем скобки и сгруппируем слагаемые с давлением $\Delta(dP_{in}^n)$ и все остальные. Тогда
\begin{eqnarray}
\label{formula620}
A=(\omega_{N-1} \cdot \beta_{N-1} \cdot \beta_{N-2} \dots \beta_1 + \omega_{N-2} \cdot \beta_{N-2} \dots \beta_1 + \dots + \nonumber ~\\
+ \omega_1 \cdot \beta_1 + \omega_0) \cdot \Delta(dP_{in}^n) + (u_{N-1} \cdot \beta_{N-1} \cdot \beta_{N-2} \dots \beta_1 + \nonumber ~\\
+ u_{N-2} \cdot \beta_{N-2} \dots \beta_1 + \dots + u_1 \cdot \beta_1 + u_0).
\end{eqnarray}

Подставляя~\eqref{formula620} в~\eqref{formula618}, получим окончательно для первой расчётной ячейки канала выражение
\begin{equation}
\label{formula621}
\Delta(dP_0^n)=S_0^0 + S_0^1 \cdot \Delta(dP_{in}^n) + S_0^2 \cdot \Delta(dP_{out}^n),
\end{equation}  
где $S_0^0 = u_{N-1} \cdot \beta_{N-1} \cdot \beta_{N-2} \dots \beta_1 + u_{N-2} \cdot \beta_{N-2} \dots \beta_1 + \dots + u_1 \cdot \beta_1 + u_0$;

\noindent $S_0^1 = \omega_{N-1} \cdot \beta_{N-1} \cdot \beta_{N-2} \dots \beta_1 + \omega_{N-2} \cdot \beta_{N-2} \dots \beta_1 + \dots + \omega_1 \cdot \beta_1 + \omega_0$;

\noindent $S_0^2 = \beta_N \cdot \beta_{N-1} \cdot \beta_{N-2} \dots \beta_1$. 

Для последней расчётной ячейки канала получим из первого уравнения~\eqref{formula617}
\begin{eqnarray}
\label{formula622}
\Delta(dP_{N-1}^n)=\beta_N \cdot \Delta(dP_{out}^n) + \alpha_N = \beta_N \cdot \Delta(dP_{out}^n) + (\omega_{N-1} \cdot \Delta(dP_{in}^n) + u_{N-1}) = \nonumber ~\\
= S_{N-1}^0 + S_{N-1}^1 \cdot \Delta(dP_{in}^n) + S_{N-1}^2 \cdot \Delta(dP_{out}^n),
\end{eqnarray}
где $S_{N-1}^0 = u_{N-1}$; $S_{N-1}^1 = \omega_{N-1}$; $S_{N-1}^2 = \beta_N$. 

Аналогично можно получить общее выражение для произвольной $j$-й ячейки. Оно будет иметь вид
\begin{equation}
\label{formula623}
\Delta(dP_j^n)=S_j^0 + S_j^1 \cdot \Delta(dP_{in}^n) + S_j^2 \cdot \Delta(dP_{out}^n),
\end{equation} 
где $S_j^0 = u_{N-1} \cdot \beta_{N-1} \cdot \beta_{N-2} \dots \beta_{j+1} + u_{N-2} \cdot \beta_{N-2} \dots \beta_{j+1} + \dots + u_{j+1} \cdot \beta_{j+1} + u_j$;

\noindent $S_j^1 = \omega_{N-1} \cdot \beta_{N-1} \cdot \beta_{N-2} \dots \beta_{j+1} + \omega_{N-2} \cdot \beta_{N-2} \dots \beta_{j+1} + \dots + \omega_{j+1} \cdot \beta_{j+1} + \omega_j$;

\noindent $S_j^2 = \beta_N \cdot \beta_{N-1} \cdot \beta_{N-2} \dots \beta_{j+1}$.

Можно усмотреть некую аналогию вычисления коэффициентов $S_j$ и метода прогонки. Действительно, сначала при прямом ходе вычисляются коэффициенты $\omega_j$ и $u_j$, а затем при обратном ходе находятся коэффициенты $S_j$. При этом справедливы следующие рекуррентные соотношения:
\begin{eqnarray}
\label{formula624}
S_{N-1}^0 = u_{N-1}; S_j^0 = u_j + S_{j+1}^0 \cdot \beta_{j+1}; \nonumber ~\\
S_{N-1}^1 = \omega_{N-1}; S_j^1 = \omega_j + S_{j+1}^1 \cdot \beta_{j+1}; \nonumber ~\\
S_{N-1}^2 = \beta_N; S_j^2 = S_{j+1}^2 \cdot \beta_{j+1}.
\end{eqnarray}

















