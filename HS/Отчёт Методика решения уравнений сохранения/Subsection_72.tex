
\label{sec:subsection72}
Сделаем вывод, по сути аналогичный проведённому в подразделе~\ref{sec:subsection62}. Рассмотрим уравнение~\eqref{formula71} применительно к внутреннему узлу. В этом случае вместо одного входящего и выходящего расхода будет рассматриваться сумма по входам и выходам узла:
\begin{align}
\label{formula77}
\rho\cdot V\cdot (a_k^h+b_k^h\cdot dh_k^n)=\sum_{j=1}^{N_{in}} \mu_j \cdot G_j^{n+1} \cdot (h_j^{n+1}-h_k^{n+1}) - \notag \\
-\sum_{j=1}^{N_{out}} (1-\mu_j)\cdot G_j^{n+1} \cdot (h_j^{n+1}-h_k^{n+1})  
+ V\cdot\frac{\partial P}{\partial\tau} + Q.
\end{align}

В правой части распишем энтальпии через значение на предыдущем шаге по времени и приращение:
\begin{align}
\label{formula78}
\rho\cdot V\cdot (a_k^h+b_k^h\cdot dh_k^n)=\sum_{j=1}^{N_{in}} \mu_j \cdot G_j^{n+1} \cdot (h_j^n+dh_j^n-h_k^n-dh_k^n) - \notag \\
-\sum_{j=1}^{N_{out}} (1-\mu_j)\cdot G_j^{n+1} \cdot (h_j^n+dh_j^n-h_k^n-dh_k^n)  
+ V\cdot\frac{\partial P}{\partial\tau} + Q.
\end{align}

Сгруппируем множители перед приращениями энтальпии в узле и в граничных ячейках. Получим
\begin{align}
\label{formula79}
-\sum_{j=1}^{N_{in}} \mu_j \cdot G_j^{n+1} \cdot dh_j^n + \notag \\
+ \left[\rho\cdot V\cdot b_k^h + \sum_{j=1}^{N_{in}} \mu_j \cdot G_j^{n+1} - \sum_{j=1}^{N_{out}} (1-\mu_j)\cdot G_j^{n+1} - min\left(\frac{\partial Q}{\partial h_k^n},0 \right) \right] \cdot dh_k^n + \notag \\
+ \sum_{j=1}^{N_{out}} (1-\mu_j)\cdot G_j^{n+1} \cdot dh_j^n + \bigg[\rho\cdot V\cdot a_k^h-\sum_{j=1}^{N_{in}} \mu_j \cdot G_j^{n+1} \cdot (h_j^n-h_k^n) + \notag \\
+ \sum_{j=1}^{N_{out}} (1-\mu_j)\cdot G_j^{n+1} \cdot (h_j^n-h_k^n)-V\cdot\frac{\partial P}{\partial\tau} - Q \bigg] = 0. 
\end{align}  

Запишем это уравнение в сокращённом виде:
\begin{equation}
\label{formula710}
\sum_{j=1}^{N_{in}} A_j^h \cdot dh_j^n + B_k^h \cdot dh_k^n + \sum_{j=1}^{N_{out}} C_j^h \cdot dh_j^n + D_k^h = 0 = F_k^h, 
\end{equation} 
где $A_j^h = -\mu_j \cdot G_j^{n+1}$; $C_j^h = (1-\mu_j)\cdot G_j^{n+1}$;

\noindent $B_k^h=\rho\cdot V\cdot b_k^h - \sum_{j=1}^{N_{in}} A_j^h - \sum_{j=1}^{N_{out}} C_j^h - min\left(\frac{\partial Q}{\partial h_k^n},0 \right)$;

\noindent $D_k^h=\rho\cdot V\cdot a_k^h+\sum_{j=1}^{N_{in}} A_j^h \cdot (h_j^n-h_k^n) + \sum_{j=1}^{N_{out}} C_j^h \cdot (h_j^n-h_k^n)-V\cdot\frac{\partial P}{\partial\tau} - Q$.

\vspace{12pt}
Вновь, как и раньше, перейдём к приращениям отклонений на текущей итерации. Получим:
\begin{equation}
\label{formula711}
\sum_{j=1}^{N_{in}} A_j^h \cdot \Delta(dh_j^n) + B_k^h \cdot \Delta(dh_k^n) + \sum_{j=1}^{N_{out}} C_j^h \cdot \Delta(dh_j^n) = -F_k^h. 
\end{equation}

Подставим в полученное уравнение вместо энтальпий в последних ячейках входящих рёбер и первых ячейках выходящих рёбер их выражения через давления в узлах согласно уравнению~\eqref{formula76}. Тогда будем иметь:
\begin{align}
\label{formula712}
\sum_{j=1}^{N_{in}} A_j^h \cdot \left((S_{N-1}^0)_j+(S_{N-1}^1)_j \cdot \Delta(dh_j^n) + (S_{N-1}^2)_j \cdot \Delta(dh_k^n)\right) + \notag \\
+ B_k^h \cdot \Delta(dh_k^n) + \sum_{j=1}^{N_{out}} C_j^h \cdot \left((S_0^0)_j+(S_0^1)_j \cdot \Delta(dh_k^n) + (S_0^2)_j \cdot \Delta(dh_j^n)    \right) = -F_k^h. 
\end{align}

Сгруппируем слагаемые перед начальными узлами входящих рёбер, конечными узлами выходящих рёбер, и рассматриваемым узлом:
\begin{align}
\label{formula713}
\sum_{j=1}^{N_{in}} A_j^h \cdot (S_{N-1}^1)_j \cdot \Delta(dh_j^n) + \left[B_k^h + \sum_{j=1}^{N_{in}} A_j^h \cdot (S_{N-1}^2)_j + \sum_{j=1}^{N_{out}} C_j^h \cdot (S_0^1)_j \right] \cdot \Delta(dh_k^n) + \notag \\
+ \sum_{j=1}^{N_{out}} C_j^h \cdot (S_0^2)_j \cdot \Delta(dh_j^n) = -\left(F_k^h+\sum_{j=1}^{N_{in}} A_j^h \cdot (S_{N-1}^0)_j + \sum_{j=1}^{N_{out}} C_j^h \cdot (S_0^0)_j\right).  
\end{align}

Перепишем~\eqref{formula713} в следующем виде:
\begin{equation}
\label{formula714}
\sum_{j=1}^{N_{in}} (A_j^h)^{*} \cdot \Delta(dh_j^n) + (B_k^h)^{*} \cdot \Delta(dh_k^n) + \sum_{j=1}^{N_{out}} (C_j^h)^{*} \cdot \Delta(dh_j^n) = -(F_k^h)^{*}, 
\end{equation}
где $(A_j^h)^{*} = A_j^h \cdot (S_{N-1}^1)_j$; $(B_k^h)^{*} = B_k^h + \sum_{j=1}^{N_{in}} A_j^h \cdot (S_{N-1}^2)_j + \sum_{j=1}^{N_{out}} C_j^h \cdot (S_0^1)_j$;

\noindent $(C_j^h)^{*} = C_j^h \cdot (S_0^2)_j$; $(F_k^h)^{*} = F_k^h+\sum_{j=1}^{N_{in}} A_j^h \cdot (S_{N-1}^0)_j + \sum_{j=1}^{N_{out}} C_j^h \cdot (S_0^0)_j$.

\vspace{12pt}
В итоге получаем систему уравнений относительно энтальпий в узлах, решая которую каким-либо методом, найдём приращения отклонений энтальпий во всех внутренних узлах рассматриваемой схемы на текущей итерации.  







