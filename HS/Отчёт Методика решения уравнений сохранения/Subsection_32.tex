
В алгоритме расчёта используется прогнозирование значений на следующем временном слое по найденным значениям в предыдущие моменты времени. Получим некоторые общие выражения метода прогнозирования.

Предположим, что значение функции на следующем шаге по времени можно выразить через значения на нескольких предыдущих шагах в следующем виде:
\begin{equation}
\label{formula322}
x(t_{n+1})=x(t_n)+\sum_{i=0}^{k}\beta_i\cdot x(t_{n-i}).
\end{equation}

Разложим функцию $x(t_{n+1})$ в ряд Тейлора в окрестности точки $t=t_{n+1}$. Получим:
\begin{equation}
\label{formula323}
x(t_{n-i})=\sum_{j=0}^{\infty}\frac{x^{(j)}(t_{n+1})}{j!}\cdot \left(\frac{t_{n-i}-t_{n+1}}{h} \right)^j \cdot h^j,
\end{equation}
где $h$ может быть вообще говоря любым числом, но в данном случае мы будем считать, что $h$ --- текущий шаг по времени, то есть $h=t_{n+1}-t_n$.

Подставим \eqref{formula323} в \eqref{formula322}, поменяв в нём местами временные отметки. Тогда будем иметь
\begin{equation}
\label{formula324}
x(t_{n+1})=x(t_n)+\sum_{i=0}^{k}\beta_i\cdot \sum_{j=0}^{\infty} (-1)^j \cdot \frac{x^{(j)}(t_{n+1})}{j!}\cdot \left(\frac{t_{n+1}-t_{n-i}}{h} \right)^j \cdot h^j.
\end{equation}

Выпишем несколько первых слагаемых двойной суммы в правой части:
\begin{align}
\label{formula325}
\sum_{i=0}^{k}\beta_i\cdot \sum_{j=0}^{\infty} (-1)^j \cdot \frac{x^{(j)}(t_{n+1})}{j!}\cdot \left(\frac{t_{n+1}-t_{n-i}}{h} \right)^j \cdot h^j = \notag \\
= \beta_0\cdot \left\{1\cdot x(t_{n+1}) -1\cdot \frac{x'(t_{n+1})}{1!} \left(\frac{t_{n+1}-t_n}{h}\right)^1 h^1 +1\cdot \frac{x''(t_{n+1})}{2!} \left(\frac{t_{n+1}-t_n}{h}\right)^2 h^2-\dots \right\}+ \notag \\
+ \beta_1\cdot \left\{1\cdot x(t_{n+1}) -1\cdot \frac{x'(t_{n+1})}{1!} \left(\frac{t_{n+1}-t_{n-1}}{h}\right)^1 h^1 +1\cdot \frac{x''(t_{n+1})}{2!} \left(\frac{t_{n+1}-t_{n-1}}{h}\right)^2 h^2-\dots \right\}+ \notag \\
+ \beta_2\cdot \left\{1\cdot x(t_{n+1}) -1\cdot \frac{x'(t_{n+1})}{1!} \left(\frac{t_{n+1}-t_{n-2}}{h}\right)^1 h^1 +1\cdot \frac{x''(t_{n+1})}{2!} \left(\frac{t_{n+1}-t_{n-2}}{h}\right)^2 h^2-\dots \right\}+ \notag \\
+ \dots	
\end{align}

Перегруппируем члены, выделив множители перед $x(t_{n+1})$, $x'(t_{n+1})$, $x''(t_{n+1})$ и так далее. Получим:
\begin{align}
\label{formula326}
&\sum_{i=0}^{k}\beta_i\cdot \sum_{j=0}^{\infty} (-1)^j \cdot \frac{x^{(j)}(t_{n+1})}{j!}\cdot \left(\frac{t_{n+1}-t_{n-i}}{h} \right)^j \cdot h^j = \notag \\
&= x(t_{n+1})\cdot \left[\beta_0+\beta_1+\beta_2+\dots \right] + \\
&+ x'(t_{n+1})\cdot \left[-\beta_0 \cdot \left(\frac{t_{n+1}-t_n}{h}\right)^1 \cdot h^1 -\beta_1 \cdot \left(\frac{t_{n+1}-t_{n-1}}{h}\right)^1 \cdot h^1 -\beta_2 \cdot \left(\frac{t_{n+1}-t_{n-2}}{h}\right)^1 \cdot h^1 -\dots \right]	+ \notag \\
&+ x''(t_{n+1}) \left[\beta_0 \cdot \left(\frac{t_{n+1}-t_n}{h}\right)^2 \cdot \frac{h^2}{2!} + \beta_1 \cdot \left(\frac{t_{n+1}-t_{n-1}}{h}\right)^2 \cdot \frac{h^2}{2!} + \beta_2 \cdot \left(\frac{t_{n+1}-t_{n-2}}{h}\right)^2 \cdot \frac{h^2}{2!} + \dots \right] + \dots \notag
\end{align}

Разложим $x(t_n)$ в ряд Тейлора в окрестности точки $t_{n+1}$:
\begin{equation}
\label{formula327}
x(t_n)=x(t_{n+1})+\frac{x'(t_{n+1})}{1!}\cdot (-h) + \frac{x''(t_{n+1})}{2!}\cdot (-h)^2 + \frac{x'''(t_{n+1})}{3!}\cdot (-h)^3 + \dots
\end{equation}

Подставим выражения \eqref{formula326} и \eqref{formula327} в \eqref{formula324}. Получим следующее уравнение:
\begin{align}
\label{formula328}
&x(t_{n+1})=x(t_{n+1})\cdot \left[1+\beta_0+\beta_1+\beta_2+\dots \right] - \notag \\
&- x'(t_{n+1})\cdot h \cdot \left[1 + \beta_0 \cdot \left(\frac{t_{n+1}-t_n}{h}\right)^1 +\beta_1 \cdot \left(\frac{t_{n+1}-t_{n-1}}{h}\right)^1 +\beta_2 \cdot \left(\frac{t_{n+1}-t_{n-2}}{h}\right)^1 +\dots \right] + \notag \\
&+ x''(t_{n+1})\cdot \frac{h^2}{2!} \cdot \left[1 + \beta_0 \cdot \left(\frac{t_{n+1}-t_n}{h}\right)^2 +\beta_1 \cdot \left(\frac{t_{n+1}-t_{n-1}}{h}\right)^2 +\beta_2 \cdot \left(\frac{t_{n+1}-t_{n-2}}{h}\right)^2 +\dots \right] -\dots
\end{align} 

Сравнивая левую и правую части, получим следующую систему уравнений для нахождения неизвестных $\beta_i$:
\begin{equation}
\label{formula329}
\left\{
\begin{aligned}
1 & = 1+\beta_0+\beta_1+\beta_2+\dots  \\
0 & = 1 + \beta_0 \cdot \left(\frac{t_{n+1}-t_n}{h}\right)^1 +\beta_1 \cdot \left(\frac{t_{n+1}-t_{n-1}}{h}\right)^1 +\beta_2 \cdot \left(\frac{t_{n+1}-t_{n-2}}{h}\right)^1 +\dots \\
0 & = 1 + \beta_0 \cdot \left(\frac{t_{n+1}-t_n}{h}\right)^2 +\beta_1 \cdot \left(\frac{t_{n+1}-t_{n-1}}{h}\right)^2 +\beta_2 \cdot \left(\frac{t_{n+1}-t_{n-2}}{h}\right)^2 +\dots \\
..... & .......................................................... 
\end{aligned}
\right.
\end{equation}

С учётом того, что $\frac{t_{n+1}-t_n}{h}=1$, систему \eqref{formula329} можно представить в следующем матричном виде:
\begin{equation}
\label{formula330}
\begin{pmatrix}
1 & 1 & 1 & \dots & 1 \\
1^1 & \left(\frac{t_{n+1}-t_{n-1}}{h} \right)^1 & \left(\frac{t_{n+1}-t_{n-2}}{h} \right)^1 & \dots & \left(\frac{t_{n+1}-t_{n-k}}{h} \right)^1 \\
1^2 & \left(\frac{t_{n+1}-t_{n-1}}{h} \right)^2 & \left(\frac{t_{n+1}-t_{n-2}}{h} \right)^2 & \dots & \left(\frac{t_{n+1}-t_{n-k}}{h} \right)^2 \\
\dots & \dots & \dots & \dots & \dots \\
1^k & \left(\frac{t_{n+1}-t_{n-1}}{h} \right)^k & \left(\frac{t_{n+1}-t_{n-2}}{h} \right)^k & \dots & \left(\frac{t_{n+1}-t_{n-k}}{h} \right)^k
\end{pmatrix} \times 
\begin{pmatrix}
\beta_0 \\
\beta_1 \\
\beta_2 \\
\dots \\
\beta_k
\end{pmatrix} =
\begin{pmatrix}
0 \\
-1 \\
-1 \\
\dots \\
-1
\end{pmatrix}.
\end{equation}

Рассмотрим исходное уравнение \eqref{formula322}:
\begin{equation*}
x(t_{n+1})=x(t_n)+\beta_0 x(t_n) + \beta_1 x(t_{n-1}) + \beta_2 x(t_{n-2}) + \dots = (1+\beta_0) x(t_n) + \beta_1 x(t_{n-1}) + \beta_2 x(t_{n-2}) + \dots
\end{equation*}

Обозначим $1+\beta_0=\beta_0^{*}$. Тогда $\beta_0=\beta_0^{*}-1$. Подставим $\beta_0^{*}$ вместо $\beta_0$ в систему \eqref{formula329}. Тогда получим:
\begin{equation}
\label{formula331}
\left\{
\begin{aligned}
1 & = 1+\beta_0^{*}-1+\beta_1+\beta_2+\dots  \\
0 & = 1 + (\beta_0^{*}-1)\cdot 1^1 +\beta_1 \cdot \left(\frac{t_{n+1}-t_{n-1}}{h}\right)^1 +\beta_2 \cdot \left(\frac{t_{n+1}-t_{n-2}}{h}\right)^1 +\dots \\
0 & = 1 + (\beta_0^{*}-1)\cdot 1^2 +\beta_1 \cdot \left(\frac{t_{n+1}-t_{n-1}}{h}\right)^2 +\beta_2 \cdot \left(\frac{t_{n+1}-t_{n-2}}{h}\right)^2 +\dots \\
. & .......................................................... 
\end{aligned}
\right.
\end{equation}

Получается, что система \eqref{formula330} оказывается эквивалентной следующей системе:
\begin{equation}
\label{formula332}
\begin{pmatrix}
1 & 1 & 1 & \dots & 1 \\
1^1 & \left(\frac{t_{n+1}-t_{n-1}}{h} \right)^1 & \left(\frac{t_{n+1}-t_{n-2}}{h} \right)^1 & \dots & \left(\frac{t_{n+1}-t_{n-k}}{h} \right)^1 \\
1^2 & \left(\frac{t_{n+1}-t_{n-1}}{h} \right)^2 & \left(\frac{t_{n+1}-t_{n-2}}{h} \right)^2 & \dots & \left(\frac{t_{n+1}-t_{n-k}}{h} \right)^2 \\
\dots & \dots & \dots & \dots & \dots \\
1^k & \left(\frac{t_{n+1}-t_{n-1}}{h} \right)^k & \left(\frac{t_{n+1}-t_{n-2}}{h} \right)^k & \dots & \left(\frac{t_{n+1}-t_{n-k}}{h} \right)^k
\end{pmatrix} \times 
\begin{pmatrix}
\beta_0^{*} \\
\beta_1 \\
\beta_2 \\
\dots \\
\beta_k
\end{pmatrix} =
\begin{pmatrix}
1 \\
0 \\
0 \\
\dots \\
0
\end{pmatrix}.
\end{equation}

Матрица полученной системы похожа на матрицу с определителем Вандермонда. Рассмотрим вопрос о нахождении значения функции на следующем шаге по времени через приращения функции на предыдущих шагах по времени. 

Выразим $x(t_{n-i})$ через восходящие разности:
\begin{equation}
\label{formula333}
\left\{
\begin{aligned}
x(t_{n-k+1}) & = x(t_{n-k}) + dX(t_{n-k}) \\
x(t_{n-k+2}) & = x(t_{n-k}) + dX(t_{n-k}) + dX(t_{n-k+1}) \\
..... & .......................................................... \\
x(t_{n-1}) & = x(t_{n-k}) + dX(t_{n-k}) + \dots + dX(t_{n-3}) + dX(t_{n-2}) \\
x(t_n) & = x(t_{n-k}) + dX(t_{n-k}) + \dots + dX(t_{n-2}) + dX(t_{n-1})
\end{aligned}
\right.
\end{equation}

Подставим $x(t_{n-i})$ из \eqref{formula333} в \eqref{formula322}: 
\begin{align}
\label{formula334}
x(t_{n+1})=x(t_n)+\beta_0\cdot [x(t_{n-k}) + dX(t_{n-k}) + \dots + dX(t_{n-2}) + dX(t_{n-1}) ] + \notag \\
+ \beta_1\cdot [x(t_{n-k}) + dX(t_{n-k}) + \dots + dX(t_{n-3}) + dX(t_{n-2})] + \dots \notag \\
+ \beta_{k-1} \cdot [x(t_{n-k}) + dX(t_{n-k})] + \beta_k \cdot x(t_{n-k}).
\end{align}

Перегруппируем слагаемые:
\begin{align}
\label{formula335}
x(t_{n+1})=x(t_n)+\beta_0 \cdot dX(t_{n-1}) + (\beta_0+\beta_1) \cdot dX(t_{n-2}) + \dots + \notag \\
+ (\beta_0 + \beta_1 + \dots + \beta_{k-1}) \cdot dX(t_{n-k}) + (\beta_0 + \beta_1 + \dots + \beta_{k-1} + \beta_k) \cdot x(t_{n-k}).
\end{align}

Согласно первому уравнению \eqref{formula330} $\beta_0 + \beta_1 + \dots + \beta_{k-1} + \beta_k = 0$, поэтому член с $x(t_{n-k})$ отбрасываем. Заменим $\beta_0$ на $\beta_0^{*}-1$. Тогда получим:
\begin{align}
\label{formula336}
x(t_{n+1})=x(t_n)+(-1+\beta_0^{*}) \cdot dX(t_{n-1}) + (-1+\beta_0^{*}+\beta_1) \cdot dX(t_{n-2}) + \dots + \notag \\
+ (-1+\beta_0^{*} + \beta_1 + \dots + \beta_{k-1}) \cdot dX(t_{n-k}) = x(t_n) + \sum_{i=1}^{k} \beta'_{i-1} \cdot dX(t_{n-i}),
\end{align} 
где 
\begin{equation}
\label{formula337}
\beta'_{i-1}=-1+\sum_{j=1}^{i} \beta_{j-1},
\end{equation} 
а вместо $\beta_0$ используется $\beta_0^{*}$.





















