
Найдём конечно-разностный аналог уравнения сохранения энергии. Для этого рассмотрим уравнение для контрольного объёма (ячейки либо узла) в виде~\eqref{formula237}. Заменим производные энтальпии и давления по формуле дифференцирования назад. Перепишем сумму, содержащую сумму расходов в гидравлических связях, следующим образом:
\begin{equation}
\label{formula414}
\sum_{k=1}^{N_{gc}} \mu_k \cdot G_k \cdot (h_k-h_j) = \sum_{k=1}^{N_{in}} \mu_k \cdot G_k \cdot (h_k-h_j) - \sum_{k=1}^{N_{out}} (1-\mu_k) \cdot G_k \cdot (h_k-h_j).
\end{equation}

Таким образом, если выходящий расход больше нуля, то во второй сумме $\mu_k=1$ и соответствующее слагаемое обнуляется, а если выходящий расход меньше нуля, то он становится входящим и вносит свой вклад в общую сумму.

Расходы запишем через сумму расходов на предыдущем шаге по времени и соответствующей восходящей разности. Кроме того, отнесём источник массы $R_m$ в соответствующую сумму расходов (в зависимости от его знака). Сумму всех входящих в контрольный объём тепловых потоков для упрощения запишем в виде $$V\cdot Q + Q_{wall} + Q_{ax} = Q+min\left(\frac{\partial Q}{\partial h_j^n},0 \right)\cdot dh_j^n.$$ В итоге получим: 
\begin{align}
\label{formula415}
&\rho\cdot V\cdot (a_j^h + b_j^h \cdot dh_j^n)=\sum_{k=1}^{N_{in}} \mu_k \cdot (G_k^n+dG_k^n) \cdot (h_k-h_j) - \nonumber ~\\
- &\sum_{k=1}^{N_{out}} (1-\mu_k) \cdot (G_k^n+dG_k^n) \cdot (h_k-h_j) + V\cdot (a_j^P+b_j^P\cdot dP_j^n)
+Q+min\left(\frac{\partial Q}{\partial h_j^n},0 \right)\cdot dh_j^n.
\end{align}

Перегруппируем слагаемые следующим образом:
\begin{eqnarray}
\label{formula416}
-\sum_{k=1}^{N_{in}} \mu_k \cdot (h_k-h_j) \cdot dG_k^n + \left(\rho\cdot V\cdot b_j^h - min\left(\frac{\partial Q}{\partial h_j^n},0 \right) \right) \cdot dh_j^n + \nonumber ~\\
+ \sum_{k=1}^{N_{out}} (1-\mu_k) \cdot (h_k-h_j) \cdot dG_k^n - V \cdot b_j^P \cdot dP_j^n + \bigg(\rho\cdot V\cdot a_j^h - \nonumber ~\\
- \sum_{k=1}^{N_{in}} \mu_k \cdot G_k^n \cdot (h_k-h_j) + \sum_{k=1}^{N_{out}} (1-\mu_k) \cdot G_k^n \cdot (h_k-h_j) - V \cdot a_j^P -Q \bigg)=0.  
\end{eqnarray}

Получим окончательно конечно-разностное уравнение сохранения энергии для контрольного объёма в виде
\begin{equation}
\label{formula417}
\boxed{\sum_{k=1}^{N_{in}} A_{kj}^h \cdot dG_k^n + B_j^h \cdot dh_j^n + \sum_{k=1}^{N_{out}} C_{kj}^h \cdot dG_k^n + D_j^h \cdot dP_j^n + E_j^h = 0 = F_j^h},
\end{equation}
где $A_{kj}^h=-\mu_k\cdot (h_k-h_j)$; $B_j^h=\rho\cdot V\cdot b_j^h - min\left(\frac{\partial Q}{\partial h_j^n},0 \right)$; $C_{kj}^h=(1-\mu_k)\cdot (h_k-h_j)$;
$D_j^h=-V \cdot b_j^P$; 

\noindent $E_j^h=\rho\cdot V\cdot a_j^h -\sum_{k=1}^{N_{in}} \mu_k \cdot G_k^n \cdot (h_k-h_j) + \sum_{k=1}^{N_{out}} (1-\mu_k) \cdot G_k^n \cdot (h_k-h_j) - V \cdot a_j^P -Q$.

В полученные дискретных аналогах~\eqref{formula44}, \eqref{formula413} и \eqref{formula417} член $F_j$ равен разности левой и правой части соответствующего уравнения сохранения.  











