
\section{Заключение}
\label{sec:conclusion}
В отчёте описаны основные решаемые в теплогидравлическом коде SimInTech уравнения сохранения для жидкости, приведён подробный вывод конечно-разностных уравнений, описано приведение их к виду, пригодному для решения итерационным методом Нью\-то\-на-Рафсона.

Приведён вывод основных уравнений для определения полей давления и энтальпии в разветвлённом теплогидравлическом контуре произвольной топологии. Описан алгоритм основного метода, осуществляющего выполнение очередного шага по времени при расчёте теплогидравлической схемы. 

В будущем планируется расширять функциональные возможности кода, в частности рассмотреть и другие численные схемы решения системы уравнений сохранения (явная, неявная), дать возможность выбора максимального числа итераций (ноль итераций соответствует безытерациооной схеме) и порядка формулы дифференцирования назад. Для каждой расчётной ячейки будет добавлено вычисление интегрального небаланса массы и энергии в ячейке в процессе расчёта и, возможно, численная компенсация этого небаланса. 

\newpage




